%
%  untitled
%
%  Created by leobessa on 2009-06-21.
%  Copyright (c) 2009 __MyCompanyName__. All rights reserved.
%
\documentclass[]{article}

% Use utf-8 encoding for foreign characters
\usepackage[utf8]{inputenc}

% Setup for fullpage use
\usepackage{fullpage}

% Uncomment some of the following if you use the features
%
% Running Headers and footers
%\usepackage{fancyhdr}

% Multipart figures
%\usepackage{subfigure}

% More symbols
%\usepackage{amsmath}
%\usepackage{amssymb}
%\usepackage{latexsym}

% Surround parts of graphics with box
\usepackage{boxedminipage}

% Package for including code in the document
\usepackage{listings}

% If you want to generate a toc for each chapter (use with book)
\usepackage{minitoc}

% This is now the recommended way for checking for PDFLaTeX:
\usepackage{ifpdf}

%\newif\ifpdf
%\ifx\pdfoutput\undefined
%\pdffalse % we are not running PDFLaTeX
%\else
%\pdfoutput=1 % we are running PDFLaTeX
%\pdftrue
%\fi

\ifpdf
\usepackage[pdftex]{graphicx}
\else
\usepackage{graphicx}
\fi
\title{Plano do Projeto de Formatura}
\author{Allan Douglas, Leonardo Bessa e Thiago Andrade  }

\date{23/06/2009}

\begin{document}

\ifpdf
\DeclareGraphicsExtensions{.pdf, .jpg, .tif}
\else
\DeclareGraphicsExtensions{.eps, .jpg}
\fi

\maketitle

%\begin{abstract}\end{abstract}

\section{Objetivo} % (fold)
\label{sec:objetivo}
% Apresentar de forma precisa e concisa o objetivo do projeto.

 O principal objetivo deste projeto é criar um sistema de recomendação baseado em recursos disponíveis na Web, que possibilite a sugestão de itens confiáveis e relevantes ao usuário.

% section objetivo (end)

\section{Justificativa} % (fold)
\label{sec:objetivos_e_justificativas}

Sistemas de Recomendação sugerem aos usuários itens que eles possam gostar baseado no comportamento prévio do usuário. Ao oferecer valor aos usuários, fazendo suposições pertinentes sobre o tipo de objetos em que estão interessados, é possível conquistar a confiança deles. A vantagem para os usuários é a facilidade de encontrar a informação sem ter a árdua tarefa de procurá-la.

As redes sociais online têm modificado a forma com que as empresas utilizam a comunicação para o comércio. Pessoas estão utilizando a Web para encontrar outras pessoas com interesses similares, fazer compras de forma mais eficiente, aprender sobre produtos e serviços e reclamar sobre produtos malfeitos e serviços pobres.

A Web está rapidamente se tornando a mídia mais importante para o marketing. A tendência é que as pessoas, cada vez mais bloqueiem os anúncios indesejados e queiram ter a capacidade de encontrar os produtos relevantes no momento adequado. É nesse contexto que surge a necessidade de uma plataforma que facilite a colaboração e que permita a criação e classificação de conteúdo pelos consumidores, de forma a permitir uma escolha mais inteligente dos melhores produtos e serviços e ao mesmo tempo criando uma mecanismo de feedback para as empresas interessadas.

Devido à grande variedade atual de produtos e serviços, as pessoas têm cada vez mais dificuldade nas suas escolhas e na argumentação sobre a possível decisão. Quanto mais produtos similares de fabricantes diferentes ou serviços realizados por diversos prestadores, mais as pessoas se vêem desnorteadas e sem saber se a decisão realizada foi a mais correta.

% section objetivos_e_justificativas (end)


\section{Estrutura Analítica do Projeto} % (fold)
\label{sec:estrutura_analitica_do_projeto}

\begin{enumerate}
		
	\item Pesquisa Bibliográfica / Monografia
	\begin{enumerate}
		\item Sistema de Recomendação
		\begin{enumerate}
			\item Baseados em Conteúdo
			\item Filtragem Colaborativa
			\item Baseados em Confiança
		\end{enumerate}
		\item Web Semântica
		\begin{enumerate}
		  \item Microformatos
			\item Resource Description Framework (RDF)
			\item Web Ontology Language (OWL)
		\end{enumerate}
		\item Web Social
		\begin{enumerate}
		  \item Padrões de Projeto Sociais
		\end{enumerate}
	\end{enumerate}
		
	\item Desenvolvimento
	\begin{enumerate}
		\item Sistema de recomendação
		\begin{enumerate}
			\item Captura e Parsing de documentos
			\item Algoritmo de recomendação
			\item Interface de consulta de recomendações
		\end{enumerate}
		\item Base de documentos semânticos
		\item Front-End
		\begin{enumerate}
			\item Aplicação para rede social
			\item Camada de Apresentação
			\item Integração com a base de documentos
			\item Integração com o Sistema de recomendação
		\end{enumerate}
	\end{enumerate}
	
\end{enumerate}


% section estrutura_analítica_do_projeto (end)

\section{Cronograma} % (fold)
\label{sec:cronograma}

\begin{enumerate}
  \item 13 de Julho de 2009
  \begin{enumerate}
    \item Entrega das seções sobre RDF e Microformatos
    \item Entrega do Sistema repositório de documentos Semânticos
  \end{enumerate}
  \item 27 de Julho de 2009
  \begin{enumerate}
    \item Entrega da seção sobre OWL
  \end{enumerate}
  \item 10 de Agosto de 2009
  \begin{itemize}
    \item Apresentação do Calendário do Projeto de Formatura
  \end{itemize}
  \item 24 de Agosto de 2009
  \item 7 de Setembro de 2009
  \begin{itemize}
    \item Apresentação da Evolução do Projeto e entrega de documento.
  \end{itemize}
  \item 21 de Setembro de 2009
  \item 5 de Outubro de 2009
  \item 19 de Outubro de 2009
  \item 2 de Novembro de 2009
  \begin{enumerate}
    \item Apresentação da Evolução do Projeto e entrega de documento.
    \item Entrega do Pôster e Press Release, Página Internet
  \end{enumerate}
  \item 16 de Novembro de 2009
  \item 30 de Novembro de 2009
  \begin{enumerate}
    \item Entrega da Monografia
  \end{enumerate}
  \item 8 de Dezembro de 2009
  \begin{enumerate}
    \item Apresentação e Demonstração Prática Perante Banca
  \end{enumerate}
\end{enumerate}

% section cronograma (end)


\section{Custos} % (fold)
\label{sec:custos}

\begin{itemize}
  \item[\textbf{Desenvolvimento}]
  
  Cada integrante trabalhará quatro horas por semana durante durante três meses, totalizando 144 horas de desenvolvimento. Considerando o valor da hora paga a um desenvolvedor de R\$ 15.00, o custo de desenvolvimento totaliza em R\$ 2.160,00.
  
  
\end{itemize}

% section custos (end)


\section{Estrutura Analítica de Riscos} % (fold)
\label{sec:estrutura_analitica_de_riscos}

% section estrutura_analítica_de_riscos (end)

\section{Análise SWOT} % (fold)
\label{sec:analise_swot}

\begin{itemize}
  \item Forças
  \begin{itemize}
    \item[\textbf{Conhecimento técnico}]
    
     Toda a equipe está familiarizada com a área técnica abordada na elaboração dos trabalho, sendo que muitas das técnicas de projeto e implantação são ensinadas no curso de engenharia de computação. Além disso, todos já realizaram estágios na área de desenvolvimento de software durante a sua carreira acadêmica.
    
    \item[\textbf{Orientação}] 
    
     O projeto é orientado pelo Prof. Dr. Jaime Simão Sichman que obteve formação de Livre-Docência em Inteligência Artificial. Além disso, o projeto conta com a co-orientação da Prof. Dra. Lucia Filgueiras que atua na área de Análise e Projeto de Interfaces Homem-Computador.
     
  \end{itemize}
  
  \item Fraquezas
  \begin{itemize}
    \item 
  \end{itemize}
  
  \item Oportunidades
  \begin{itemize}
    \item 
  \end{itemize}
  
  \item Ameaças
  \begin{itemize}
    \item 
  \end{itemize}
\end{itemize}


% section análise_swot (end)

\section{Gerenciamento de Configuração} % (fold)
\label{sec:gerenciamento_de_configuracao}

Todos os arquivos como documentos e código-fontes estarão sob o controle de versão usando um software SVN. Esse software controlará todas mudanças nos arquivos. Ao término de cada ciclo de desenvolvimento será feito um \emph{release} (lançamento) usando os arquivos do repositório de códigos. Os arquivos de um determinado \emph{release} serão marcados (\emph{tagging}) dentro do repositório para que as mudanças a partir de um determinado lançamento possam ser rastreadas.

Será usado também um software para controle de tarefas (\emph{tasks}), que incluirá o controle de defeitos (\emph{bug tracking}). Os arquivos serão modificados para atender às tarefas alocadas nesse \emph{software}. Como a criação de tarefas será feita a partir do EAP e a criação de \emph{bug reports} seguirá o plano de testes, todas as mudanças terão uma causa identificável e rastreável.
% section gerenciamento_de_configuracao (end)


%\bibliographystyle{plain}
%\bibliography{}
\end{document}
