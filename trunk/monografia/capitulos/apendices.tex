\appendix
\chapter{Resultados dos Testes de Hipótese}
\label{anexo_hipoteses}

Os resultados abaixo mostram os valores que validam as hipóteses apresentadas na monografia. A hipótese H0 foi de que as duas distribuições eram semelhantes, sendo que os detalhes do enunciado da hipótese de cada teste dependem do tipo de teste em questão.

Os seguintes testes foram utilizados:

\begin{itemize}
    \item T-Student~\footnote{http://en.wikipedia.org/wiki/Student\%27s\_t-test}
    \item Mann-Whitney rank test~\footnote{http://en.wikipedia.org/wiki/Mann\%E2\%80\%93Whitney\_U\_test}
    \item Wilcoxon rank-sum test~\footnote{http://en.wikipedia.org/wiki/Mann-Whitney-Wilcoxon\_test}
    \item Kruskal-Wallis H-test~\footnote{http://en.wikipedia.org/wiki/Kruskal\%E2\%80\%93Wallis\_one-way\_analysis\_of\_variance}

\end{itemize}

Nota-se que todas as hipóteses H0 foram rejeitadas com um nível de significância de 10\%, resultando na aceitação das hipóteses H1.

Os cálculos foram feitos utilizando o pacote para aplicações científicas SciPy~\footnote{http://www.scipy.org/}.

\begin{table}
\centering
\begin{tabular}{|r|c|c|}
    \hline
    \textbf{Tipo do teste} & \textbf{Valor da estatística} & \textbf{\textit{p}} \\
    \hline
T-Student (t) & 1.8483967257 & 0.0647589406833 \\
\hline 
Mann-Whitney rank test (u) & 200627.5 & 0.0482826622258 \\
\hline 
Wilcoxon rank-sum test (z) & 1.94675892865 & 0.0515636430578 \\
\hline 
Kruskal-Wallis H-test (H) & 4.01357666489 & 0.0451353055176 \\
\hline 

\end{tabular}
\caption{\it Hipótese \ref{sec:hip_amigos_melhor_desconhecidos}: Amigos recomendam melhor do que Desconhecidos (H1)}
\end{table}


\begin{table}
\centering
\begin{tabular}{|r|c|c|}
    \hline
    \textbf{Tipo do teste} & \textbf{Valor da estatística} & \textbf{\textit{p}} \\
    \hline
T-Student (t) & -10.0118428583 & 0.0 \\
\hline 
Mann-Whitney rank test (u) & 242654.0 & 0.0 \\
\hline 
Wilcoxon rank-sum test (z) & -9.31352135363 & 0.0 \\
\hline 
Kruskal-Wallis H-test (H) & 90.9919336849 & 1.44258997389e-21 \\
\hline 

    \end{tabular}
\caption{\it Teste de que recomendações Diretas são melhores aceitas que RBC (H1)}
\end{table}


\begin{table}
\centering
\begin{tabular}{|r|c|c|}
    \hline
    \textbf{Tipo do teste} & \textbf{Valor da estatística} & \textbf{\textit{p}} \\
    \hline
T-Student (t) & 3.69171586374 & 0.000234932528179 \\
\hline 
Mann-Whitney rank test (u) & 107016.0 & 0.000473211777461 \\
\hline 
Wilcoxon rank-sum test (z) & 3.44234121161 & 0.000576702325041 \\
\hline 
Kruskal-Wallis H-test (H) & 12.5985026565 & 0.000386055903711 \\
\hline 

    \end{tabular}
\caption{\it Teste de que RBC é melhor aceito que RBP (H1)}
\end{table}


\begin{table}
\centering
\begin{tabular}{|r|c|c|}
    \hline
    \textbf{Tipo do teste} & \textbf{Valor da estatística} & \textbf{\textit{p}} \\
    \hline
T-Student (t) & -2.82659786947 & 0.00479944619232 \\
\hline 
Mann-Whitney rank test (u) & 109738.5 & 0.00412473541347 \\
\hline 
Wilcoxon rank-sum test (z) & -2.83708585456 & 0.0045527367927 \\
\hline 
Kruskal-Wallis H-test (H) & 8.41106668483 & 0.00372943777944 \\
\hline 

    \end{tabular}
\caption{\it Teste de que RBI é melhor que aceito que RBC (H1)}
\end{table}

