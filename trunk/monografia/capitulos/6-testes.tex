\chapter{TESTES} % (fold)
\label{cha:testes} % (fold)

\section{Descrição do Experimento}
\label{cha:descricao_do_experimento}

 Inicialmente 12 pessoas serão convidadas a participar do experimento. Cada uma das 12 pessoas deverá convidar outras 4 pessoas para formarem um grupo de amigos. Sendo assim, formam-se 12 grupos de 5 amigos, totalizando 60 pessoas. Cada pessoa que participar de um grupo de amigos deverá concordar em ser definida como amigo de todos no grupo, ou seja, todas as pessoas pertencentes a um mesmo grupo de amigos se conhecem e se consideram amigas.

 Cada uma das 12 pessoas receberá 5 Termos de Consentimento Livre e Esclarecido (TCLE), para que seja lido e assinado por todos que concordaram em participar do grupo e, sendo assim, do experimento. Após ler e assinar o TCLE, a pessoa torna-se um participante do experimento e deve informar o seu e-mail para a pessoa que lhe convidou. De posse dos e-mails de todos presentes no grupo de amigos, cada uma das 12 pessoas deverá enviá-los aos moderadores do experimento.

 Um e-mail será enviado a cada um dos 60 participantes com um endereço da internet para que o participante possa se cadastrar. Os grupos de amigos já serão formados no sistema utilizado no experimento com base nos endereços de e-mail enviados por cada uma das 12 pessoas que foram incialmente convidadas a participar do experimento. Clicando no endereço da internet descrito no e-mail, o participante acessará a tela de cadastro no sistema, onde ele deverá informar os seguintes dados:

\begin{itemize}
	\item Nome
	\item Sexo
	\item Faixa etária (até 10 anos; 10 a 20 anos; 20 a 30 anos; 30 a 40 anos; 40 a 50 anos; 50 a 60 anos; mais de 60 anos)
	\itemFoto (Não obrigatório. Caso queira, o participante poderá escolher uma foto sua para representar o seu perfil no sistema.)
\end{itemize}

 Após isso, o participante deverá escolher a opção de salvar os seus dados e então será remetido a outra página do sistema que mostrará uma lista com 20 produtos escolhidos pelos moderadores. Tais produtos serão escolhidos com base na sua popularidade definida no site www.submarino.com. O participante deverá avaliar TODOS os 20 produtos para passar para a próxima fase do experimento.

 Avaliar um produto significa dizer o quanto o participante acha aquele produto relevante para si, ou seja, o quanto ele gosta do produto. A avaliação é feita por meio da atribuição de uma nota de 1 a 5, sendo 1 o produto ser irrelevante ou inútil ao participante e 5 o produto ser completamente relevante ou muito útil a ele. 

 Caso o participante não conheça o produto, ou seja, não tivesse ouvido falar daquele produto ou não tivesse conhecimento suficiente para reconhecê-lo até o instante em que o produto lhe foi apresentado pelo sistema, ele deve escolher a opção "Não conheço" e a opção de avaliar o produto será desabilitada, pois não é de interesse do sistema saber da opinião do participante sobre um produto que ele não tem conhecimento algum.

 Selecionando a opção "Não conheço", o sistema atribuirá nota 0 (zero) à avaliação do produto na base de dados do participante. Isso é feito porque produtos não avaliados pelo usuário contém nota de avaliação nula na base de dados e não são considerados pelo algoritmo do sistema de recomendação. Um produto que não é conhecido pelo usuário, e sendo assim não avaliado pelo mesmo, não deverá ser utilizado no algoritmo de recomendação, porém a informação de não conhecimento por parte do usuário deverá ser guardada na base de dados para posterior análise. Além disso, o usuário poderá avaliar o produto posteriormente, caso tenha procurado a respeito do mesmo.

 Terminadas as avaliações dos 20 produtos, o participante será remetido a outra página do sistema que mostrará uma listagem aleatória dos produtos cadastrados no sistema. O participante deverá procurar no sistema apenas 10 produtos de sua escolha e avaliá-los da mesma forma como feito anteriormente com os 20 produtos apresentados pelo sistema. Neste caso, o participante deverá avaliar apenas produtos de seu conhecimento, sendo que a opção de "Não conheço" será desabilitada.

 Os produtos poderão ser localizados a partir da filtragem de categorias e de um campo de busca por texto, onde o participante digita o nome do produto e o sistema o procura na base de dados, atualizando a lista com os resultados obtidos. As categorias de produtos presentes no sistema são:

\begin{itemize}
	\item Roupas
	\item Músicas
	\item Filmes
	\item Eletrônicos
	\item Livros
\end{itemize}

 Para facilidade do participante, o sistema exibirá na página o número de produtos restantes a serem avaliados.
 
 Assim que os 10 produtos forem avaliados, o participante será remetido a uma página do sistema que mostrará as fotos com nome dos 4 amigos do seu grupo e uma área com listagem aletória de produtos cadastrados no sistema. Será solicitado ao participante que ele realize 5 recomendações a cada um dos quatro amigos do seu grupo, totalizando assim 20 recomendações a serem feitas aos seus amigos. Para isso, o participante localiza um produto no sistema e escolhe a opção de recomendar a uma pessoa. Desse modo, será exibida a tela de recomendação de produto, com as informações do produto escolhido e uma lista com seus quatro amigos. O participante escolhe para quais amigos ele recomendará aquele produto e confirma a recomendação.
 
 O propósito das recomendações é que apenas boas recomendações sejam realizadas pelos participantes. Uma boa recomendação é a indicação de um produto que o participante acha que seu amigo gostará e será relevante a ele naquele momento.

 Depois de fazer as 5 recomendações a cada amigo do seu grupo, o sistema mostrará ao participante uma lista contendo foto e nome de 10 participantes que não façam parte do seu grupo e solicitará a ele que recomende apenas 1 produto a cada uma dessas pessoas.
 
 Para poder fazer boas recomendações, o participante terá acesso às avaliações de produtos feitas por estas pessoas, além das informações de cadastro de cada uma delas.

 \section{Protótipo}
 \label{cha:prototipo}

 Um protótipo do sistema foi desenvolvido para que se pudesse ter noção da usabilidade e do funcionamento do sistema de recomendação. Inicialmente as recomendações são realizadas sem levar em conta os dados de confiança e reputação entre usuários. A Figura~\ref{fig:tela_inicial_prototipo} mostra a página inicial do protótipo, contendo a listagem dos primeiros produtos presentes na base de dados e a opção do usuário fazer o \textit{login} no sistema.
 
\begin{figure}
  \centering
  \includegraphics[width=\textwidth]{imagens/Tela_Inicial_Prototipo}
  \caption{\it Tela inicial do protótipo}
  \label{fig:tela_inicial_prototipo}
\end{figure}

 Para se cadastrar, a pessoa necessita apenas informar o seu nome de usuário, e-mail e entrar com uma senha pessoal. A tela de \textit{login} solicita apenas o nome de usuário e senha. Após a validação dos dados, o sistema retorna para a tela inicial para que o usuário possa detalhar um produto de seu interesse. Há a opção de avaliar o produto sem conferir os seus detalhes. Para avaliar o produto o usuário escolhe de 1 a 5, sendo 1 não gostar do produto e 5 gostar muito, e clicar em uma das cinco estrelas. O número de estrelas coloridas mostra a nota da avaliação. 
 
 Para efetuar o \emph{login} o sistema solicita apenas o nome de usuário e senha. Após a autenticação, o sistema retorna para a tela inicial para que o usuário possa procurar um produto de seu interesse. Ao escolher um produto, o sistema abre os seus detalhes incluindo nome, foto e descrição completa, como mostra a Figura~\ref{fig:detalhe_produto_prototipo}.

\begin{figure}
  \centering
  \includegraphics[width=\textwidth]{imagens/Detalhe_Produto_Prototipo}
  \caption{\it Detalhe do produto}
  \label{fig:detalhe_produto_prototipo}
\end{figure}

 Na tela de detalhamento do produto o usuário poderá editar as informações do mesmo ao escolher a opção \textit{edit}. Também é possível visualizar os comentários (\textit{reviews}) feitos pelos usuários sobre o produto detalhado, além de fazer o seu próprio comentário, clicando em \textit{Make a review}.
  
 A encontrar um produto interessante o usuário tem a opção de recomendá-lo para outros da rede social. A opção \emph{Recommend it!} leva o usuário a tela ilustrada na Figura~\ref{fig:recomendacao_produto_prototipo}, onde é possível selecionar os amigos para os quais a recomendação será enviada.         
 
 \begin{figure}
   \centering
   \includegraphics[width=\textwidth]{imagens/TELA_RECOMENDACAO_PROTOTIPO}
   \caption{\it Tela de recomendação de produto}
   \label{fig:recomendacao_produto_prototipo}
 \end{figure}
