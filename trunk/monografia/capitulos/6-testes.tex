\chapter{TESTES} % (fold)
\label{cha:testes} % (fold)

\section{Descrição do Experimento}
\label{cha:descricao_do_experimento}

 Foi realizado um experimento com 60 pessoas que se cadastraram e formaram uma rede social disposta em 12 grupos de 5 amigos. O sistema de recomendação utiliza essa rede social para obter as informações de relações de amizade entre os usuários. Neste contexto, um usuário é considerado amigo do outro quando eles pertencem a um mesmo grupo de amigos. Os grupos de usuários foram definidos a partir do envio de um e-mail com um convite de um usuário já cadastrado no sistema a outra pessoa. Este e-mail possuía um endereço de cadastro, onde foi possível a pessoa se cadastrar e automaticamente já fazer parte do grupo de amigos do usuário que a convidou.

 Os usuários receberam recomendações dos seus amigos e de outras pessoas presentes na rede. Além disso, foram utilizadas as informações de avaliações de produtos por parte dos usuários para o sistema de recomendação poder recomendar outros produtos a eles. Foram utilizados 3 tipos de algoritmos pelo sistema de recomendação:

\begin{itemize}
	\item Baseado em similaridade entre perfis
	\item Baseado em similaridade de produtos
	\item Baseado em confiança
\end{itemize}
 
 Inicialmente 12 pessoas foram convidadas a participar do experimento. Cada uma das 12 pessoas convidou outras 4 pessoas para formarem um grupo de amigos. Sendo assim, formaram-se 12 grupos de 5 amigos, totalizando 60 pessoas. Cada participante de um grupo de amigos concorda em ser definido como amigo de todos no grupo, ou seja, todas as pessoas pertencentes a um mesmo grupo de amigos se conhecem e se consideram amigas.

 Os participantes realizaram o cadastro informando os seguintes dados pessoais:

\begin{itemize}
	\item Nome
	\item Sexo
	\item Faixa etária (18-25, 26-30, 31-40, +40)
	\item Foto
\end{itemize}

 O experimento consistia em 6 etapas onde os participantes deveriam executar tarefas descritas no próprio sistema.
 
\subsection{Primeira Etapa}
\label{cha:primeira_etapa}

 Logo após a finalização do cadastro no sistema, a primeira etapa do experimento era iniciada. Ela consistia na avaliação de 20 produtos em comum a todos os participantes de todos os grupos. Esses produtos foram selecionados previamente com o intuito de abranger os diferentes gostos de todos os participantes. Esta etapa foi necessária para que houvesse um \textit{rating overlap} no sistema, ou seja, todos os usuários avaliarem um mesmo conjunto de produtos. Com isso, foi possível calcular a similaridade de perfis entre todos os participantes da rede. Além disso, com esta etapa, resolve-se o problema dos \textit{coldstart users}, pois todos os usuários iniciam o uso do sistema ativamente.

 Avaliar um produto significa dizer o quanto o participante acha aquele produto relevante para si, ou seja, o quanto ele se interessa pelo produto. A avaliação é feita por meio da atribuição de uma nota de 1 a 5, sendo 1 o participante não ter interesse e 5 o participante ter muito interesse naquele produto. Esse interesse não necessita ser um interesse de compra. O objetivo dessa avaliação é saber realmente o interesse do participante, podendo ser cobiça ou curiosidade, além de admiração pelo produto.

%TODO inserir imagem da avaliação de produto
 
 Caso o participante não conhecesse o produto na hora de avaliá-lo, ou seja, não tivesse ouvido falar dele ou não tivesse conhecimento suficiente para reconhecê-lo, até o instante em que ele lhe foi apresentado, ele deveria informar isso ao sistema. Mesmo não conhecendo, o participante avaliava o produto de acordo com o seu grau de interesse. Neste caso, a avaliação deveria ser feita apenas com base na foto e descrição do produto.
 
\subsection{Segunda Etapa}
\label{cha:segunda_etapa}

 Assim que o participante terminava a primeira etapa, automaticamente a segunda etapa era carregada. Esta visava conhecer os interesses particulares dos participantes. Para isso, foram disponibilizados vários produtos separados nas diversas categorias:

 \begin{itemize}
	\item CDs
	\item DVDs e Blu Ray
	\item Livros
	\item Livros Importados
	\item Celulares e Telefonia Fixa
	\item Vinhos e Bebidas
	\item Relógios e Presentes
	\item Informática e Acessórios
\end{itemize} 
 
%TODO inserir imagem da busca de produtos
 
 Os dados dos produtos foram retirados do site da Submarino\footnote{http://www.submarino.com}, sendo que apenas algumas categorias de livros, CDs e DVDs foram cadastradas no sistema e apenas os 60 produtos mais vendidos das outras categorias foram considerados no cadastro.
 
 Uma busca por nome foi implementada para que os participantes pudessem localizar um produto a seu gosto e avaliá-lo. Nesta etapa também estava disponível a opção de "Não conheço", pois era possível o participante encontrar durante a busca um produto desconhecido que o agradasse.
 
\subsection{Terceira Etapa}
%TODO

 O sistema aguardará todos os participantes avaliarem os 20 produtos escolhidos pelos pesquisadores e os 10 produtos escolhidos por eles mesmos. Somente quando todos terminarem as etapas de avaliação de produtos que a etapa seguinte estará disponível. Esse sincronismo é necessitado pela próxima etapa. Neste meio tempo, o participante que já finalizou as suas tarefas receberá uma mensagem do sistema indicando que ele receberá um e-mail quando a próxima etapa estiver liberada.

\subsection{Quarta Etapa}
%TODO

 Na etapa seguinte, o sistema mostrará as fotos com o nome dos 4 amigos do grupo do participante e uma área com listagem aleatória de produtos cadastrados no sistema. Será solicitado ao participante que ele realize 5 recomendações a cada um dos quatro amigos do seu grupo, totalizando assim 20 recomendações a serem feitas. Para isso, o participante localiza um produto no sistema e escolhe a opção de recomendar a uma pessoa. Desse modo, será exibida a tela de recomendação de produto, com as informações do produto escolhido e uma lista com seus quatro amigos. O participante escolhe para quais amigos ele recomendará aquele produto e confirma a recomendação.
 
 O propósito das recomendações é que apenas boas recomendações sejam realizadas pelos participantes. Uma boa recomendação é a indicação de um produto que o participante acha que seu amigo achará interessante.

\subsection{Quinta Etapa}
%TODO

 Depois de fazer as 5 recomendações a cada amigo do seu grupo, o sistema mostrará ao participante uma lista contendo foto e nome de 10 participantes que não façam parte do seu grupo de amigos e solicitará a ele que recomende apenas 1 produto a cada uma dessas pessoas, totalizando 10 recomendações. Para poder fazer boas recomendações, o participante terá acesso às avaliações de produtos feitas por estas pessoas, além das informações de cadastro de cada uma delas. Para facilitar tal tarefa, o sistema indicará ao participantes quais das 10 pessoas ainda não receberam a sua recomendação.

 O sistema aguardará todos os participantes finalizarem as etapas de recomendar produtos aos amigos e aos desconhecidos antes de continuar o experimento. Neste meio tempo, o participante que já finalizou as suas tarefas visualizará no sistema uma mensagem indicando que ele receberá um e-mail quando a próxima etapa estiver liberada. Após a conclusão das etapas de recomendação por parte de todos os participantes, será enviado um e-mail a cada um deles avisando a continuidade do experimento.
 
 Ao entrar novamente no sistema o participante visualizará 20 recomendações de produtos. Estas são compostas de 5 recomendações feitas pelos seus amigos, 5 recomendações feitas por desconhecidos - em ambos os casos escolhidas aleatoriamente pelo sistema - e 10 recomendações feitas pelo sistema, cinco delas utilizando os algoritmos de recomendação com base em similaridade entre perfis e o restante similaridade entre produtos. O participante deverá avaliar o produto demonstrando o seu interesse na mesma escala de 1 a 5 utilizada para avaliar produtos anteriormente. A opção "Não conheço" está disponível nesta etapa. Não ficará visível ao participante quem é o autor da recomendação. Isso é feito para que não haja nenhuma influência sobre ele quando for avaliar o produto. Chamamos esta etapa de "teste cego". O sistema mostrará quantas recomendações restam para serem avaliadas pelo participante.

\subsection{Sexta Etapa}
%TODO

 Após avaliar as 20 recomendações, o participante receberá outras 30 recomendações. Neste caso serão utilizadas as 15 recomendações restantes realizadas pelos seus amigos, as 5 recomendações restantes feitas a ele por desconhecidos e 10 recomendações feitas pelo sistema utilizando o algoritmo de recomendação com base na confiança. Nesta etapa ficará visível ao participante quem é o autor da recomendação (inclusive quando for o sistema). Após terminar de avaliar todas as recomendações, o experimento será finalizado e uma mensagem de agradecimento será mostrada ao participante.
