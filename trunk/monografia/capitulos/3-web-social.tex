\chapter{WEB SOCIAL} % (fold)
\label{cha:web_social}

\section{O que é Web Social}

O conceito de Web Social está relacionado diretamente com o termo Web 2.0, termo criado em 2004 pela empresa O'Reilly Medial, o qual designa o uso da internet como plataforma para novos tipos de comunidades e serviços. Antes do advento da Web2.0, o fluxo de informação era praticamente unidirecional, sendo que os dados eram disponibilizados pelo proprietário do site da internet e apenas visualizado pelas pessoas que o frequentavam. A troca de informações era muito baixa, dado que as pessoas não podiam comentar sobre o conteúdo a elas exposto dando a sua opinião positiva ou negativa.

Com o surgimento da Web 2.0 a contribuição das pessoas tornou-se fundamental para a disponibilidade do conteúdo presente na internet. O fluxo de informações vem se tornando bidirecional, ou seja, as pessoas visualizam um conteúdo presente em um site e podem comentar, alterar ou até mesmo adicionar novos dados.

Neste cenário, as pessoas se cadastram no site e criam perfis com suas informações pessoais. Há também a possibilidade da criação de comunidades, as quais têm grande importância, pois o objetivo das pessoas que se cadastram em uma rede social é a de encontrar outras, amigas ou não, com os mesmos interesses. Isso facilita o engajamento delas e o desenvolvimento do assunto, já conhecido por todos no contexto da comunidade.

% Motivando novos usuários
\section{Qualidade da Informação}

Com a dependência das pessoas para a formação do conteúdo, as redes sociais dependem de que os novos usuários contribuam, pois não existe uma recompensa para aqueles que compartilham informações, a não ser o reconhecimento e a melhora do conteúdo do site, porque quanto mais as pessoas contribuem para a rede social, mais rica ela será.

Um dos desafios é justamente motivar os novos usuários a entrar com informações. Durante a formação de uma rede social, o princípio básico adotado é sempre o cadastro inicial de algumas pessoas influentes, para que depois estas possam enviar convites para seus amigos e conhecidos. Também há a opção de convidar amigos presentes em outras redes, o que torna mais fácil o processo de localização de pessoas conhecidas. Após isso, algumas das técnicas adotadas é mostrar aos usuários o que seus amigos estão fazendo na rede. Isso faz com que as pessoas os imitem, pois um dos comportamentos identificados em redes sociais é que as pessoas entram e primeiro observam os outros, para então tomar atitudes semelhantes.

Mesmo fazendo com que as pessoas compartilhem informações, a qualidade destas depende ainda do propósito da rede. Algumas redes sociais são apenas para diversão, como o Orkut\footnote{www.orkut.com} e o MySpace\footnote{www.myspace.com}, porém existem redes sociais voltadas para a criação de perfis profissionais, sendo o LinkedIn\footnote{www.linkedin.com} um exemplo. As informações contidas no Orkut e no MySpace não necessariamente são de alta qualidade, uma vez que muitas pessoas não utilizam sua identificação real. Já no LinkedIn, uma rede de contatos profissionais, as pessoas, em sua maioria, entram com suas informações reais, fazendo com que a qualidade desses dados seja muito mais relevante.

% Por que ela depende de recomendação?
%\section{Necessidade de Sistemas de Recomendação}



% chapter web_social (end)

% Tecnologias e conceitos empregados, contextualização do Projeto de Formatura em sua área de aplicação, revisão da literatura.