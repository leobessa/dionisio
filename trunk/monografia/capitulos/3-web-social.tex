\chapter{WEB SOCIAL} % (fold)
\label{cha:web_social}

\section{O que é Web Social}

O conceito de Web Social está relacionado diretamente com o termo Web 2.0, termo criado em 2004 pela empresa estadunidense O'Reilly Medial, o qual designa o uso da internet como plataforma para novos tipos de comunidades e serviços. Antes o fluxo de informação era praticamente unidirecional, sendo que os dados eram disponibilizados pelo proprietário do site da internet e apenas visualizado pelas pessoas que o frequentavam. A troca de informações era muito baixa, dado que as pessoas não podiam comentar sobre o conteúdo a elas exposto dando a sua opinião positiva ou negativa.

Com o surgimento da Web 2.0 a contribuição das pessoas tornou-se fundamental para a qualidade do conteúdo presente na internet. O fluxo de informações vem se tornando bidirecional, ou seja, as pessoas visualizam um conteúdo presente em um site e podem comentar, alterar ou até mesmo adicionar novos dados. Em Web Social, quanto melhor a informação disponibilizada pelas pessoas, melhor o site.

Neste cenário, as pessoas se cadastram no site e criam perfis com suas informações pessoais. Há também a possibilidade da criação de comunidades, as quais têm grande importância, pois o objetivo das pessoas que se cadastram na rede é a de encontrar outras pessoas, amigas sua ou não, com os mesmos interesses. Isso facilita o engajamento delas e o desenvolvimento do assunto, já conhecido por todos no contexto da comunidade.

% Qual é a melhor solução neste contexto?



% Por que ela depende de recomendação?

% chapter web_social (end)

% Tecnologias e conceitos empregados, contextualização do Projeto de Formatura em sua área de aplicação, revisão da literatura.