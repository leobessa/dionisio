\chapter {Termo de Consentimento Livre e Esclarecido (TCLE)}
\label{cha:TCLE}

\begin{figure}[ht]
  \centering
  \includegraphics[width=2.76cm]{imagens/minerva.png}
\end{figure}
\centerline{\textbf{Universidade de São Paulo}}
\centerline{\textbf{Escola Politécnica de Engenharia}}
\centerline{\textbf{Departamento de Computação e Sistemas Digitais}}
\vspace{0.3in}
 Convidamos o (a) Sr(a). para participar do experimento ``Dionísio: um sistema de recomendação baseado em confiança'', que tem como objetivo colher dados de avaliações de produtos através de sistemas de recomendação que utilizam redes sociais na Internet.

 Sistemas de Recomendação sugerem às pessoas itens que elas possam gostar, baseados no comportamento prévio delas fazendo suposições sobre o tipo de produtos em que elas estão interessadas. Atualmente a Internet conta com uma quantidade de informação muito grande. A vantagem do uso de sistemas de recomendação para as pessoas é a facilidade de encontrar a informação, sem ter a árdua tarefa de procurá-la. Estamos estudando uma nova forma de se recomendar produtos para as pessoas com base em parâmetros de confiança extraídos de redes sociais. Denominamos esse novo sistema de recomendação como sistemas de recomendação baseados em confiança.
	
 Pedimos a sua participação no experimento porque há a necessidade da utilização do sistema por pessoas, com informações reais, para que possamos verificar a eficiência do nosso sistema de recomendação e compará-lo com os já existentes. Dois dos algoritmos existentes que também serão utilizados no experimento levam em conta a similaridade de produtos e a similaridade entre os perfis das pessoas.

 O experimento iniciará com um cadastro solicitando as informações pessoais básicas: nome, sexo, faixa etária e foto. Tais dados serão utilizados apenas para exposição no experimento. Caso não seja da sua vontade exibir a sua foto, qualquer outro arquivo de imagem que não contenha conteúdo ofensivo aos outros participantes poderá ser utilizado. Além disso, na etapa de cadastro você poderá criar o seu login e senha para acesso. Você será inserido em um grupo contendo seus 4 amigos que também participam do experimento. O sistema contém informações de produtos extraídos do site www.submarino.com.

 Após o cadastro você deverá avaliar 30 produtos, 20 escolhidos pelo sistema e 10 a seu gosto, em uma escala de 1 (não tenho interesse neste produto) a 5 (tenho muito interesse neste produto). Esta avaliação não significa necessariamente um interesse de compra do produto, ou se você já o possui ou não. Nós queremos apenas saber se este produto lhe é interessante.

 Caso não o conheça, haverá a opção ``Não conheço'' disponível. Porém na avaliação você deverá informar se o produto lhe despertou o interesse ou não. O objetivo dessa etapa é obter informações sobre os seus interesses em relação a produtos. O sistema de recomendação precisa dessas informações para indicar produtos que provavelmente irão lhe interessar.

 Depois será solicitado que você faça recomendações de produtos a seus amigos e a pessoas presentes na rede que não fazem parte da sua equipe. Após todos os participantes terminarem esta etapa, serão mostradas recomendações de produtos feitas a você tanto por outros participantes quanto pelo sistema. Você também deverá avaliar esses produtos.

Lembre-se que não estamos avaliando você e sim o sistema de recomendação.  Todos os dados inseridos no sistema serão analisados apenas estatisticamente. Apenas um número de identificação gerado aleatoriamente pelo sistema estará relacionado aos seus dados. Não será possível aos pesquisadores identificá-lo a partir dos dados das avaliações de produtos.

	Guardaremos seus dados por pelo menos 2 (dois) anos.

	Você poderá pedir informações sobre a pesquisa a qualquer momento, durante e após a sua participação. Os endereços e telefones de contato com os pesquisadores da Escola Politécnica estão no fim desta carta.

	Finalmente, ressaltamos que sua participação é voluntária e que você não irá receber nenhuma remuneração ou prêmio pela sua participação.

Se você concordar em participar, solicitamos a assinatura no termo em anexo.

Agradecemos pela sua atenção!

Atenciosamente,

\vspace{0.2in}

Jaime Simão Sichman

\vspace{0.2in}

Coordenador da pesquisa

\vspace{0.2in}

Para esclarecimento de dúvidas:

Allan Douglas R. de Oliveira
\begin{quote}
	E-mail: allandouglas@gmail.com
\end{quote}
Leonardo Nicacio Bessa
\begin{quote}
	E-mail: leobessa@gmail.com
\end{quote}	
Thiago Rodrigues Andrade
\begin{quote}
	E-mail: thiago.rodrigues.andrade@gmail.com
\end{quote}
Jaime Simão Sichman
\begin{quote}
Escola Politécnica da Universidade de São Paulo

Avenida Professor Luciano Gualberto, travessa 3, n. 158

05508-900 – São Paulo – SP

Tel.: (11) 3091-5397

E-mail: jaime.sichman@poli.usp.br
\end{quote}
Lucia Filgueiras
\begin{quote}
Escola Politécnica da Universidade de São Paulo

Avenida Professor Luciano Gualberto, travessa 3, n. 158

05508-900 – São Paulo – SP

Tel.: (11) 3091-5200

E-mail: lucia.filgueiras@poli.usp.br
\end{quote}
\vspace{3.5in}
\centerline{\textbf{TERMO DE CONSENTIMENTO LIVRE E ESCLARECIDO}}

Eu \hspace{9.5cm}, RG

declaro que concordo em participar do experimento ``Dionísio: um sistema de recomendação baseado em confiança''.

Fui informado(a) sobre os detalhes da pesquisa conforme a carta anexa.

Eu entendo que os dados que eu inserir no sistema estarão disponíveis ao término do experimento sem nenhuma referência ao meu nome ou fotografia, sendo relacionado  apenas a um número de identificação gerado aleatoriamente pelo sistema.

Entendo que posso desistir de participar das atividades quando quiser. 

Entendo que meu nome verdadeiro ou fotografia não vão aparecer nos relatórios e trabalhos publicados sobre a pesquisa.

Entendo que não vou nenhum tipo de remuneração por participar desta pesquisa.

Declaro que, após convenientemente esclarecido pelo pesquisador e tendo entendido o que me foi explicado, consinto em participar do presente experimento.

\vspace{1in}

Assinatura do pesquisador \hspace{4cm}  Assinatura do participante

\vspace{1in}

	 
Nome do pesquisador

\vspace{1in}

Data
