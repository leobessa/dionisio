\chapter{ESPECIFICAÇÃO DO PROJETO} % (fold)
\label{cha:especificacao_do_projeto}

% Retirar web semântica e redefinir o projeto como o desenvolvimento de um sistema de recomendação com base em confiança acoplado a uma rede social.

 O projeto consiste no desenvolvimento de um sistema de recomendação baseado em confiança acoplado a uma rede social.

\section{Visão Geral do Sistema} % (fold)
\label{sec:visao_do_sistema}
 Os principais blocos do sistemas podem ser vistos na Figura~\ref{fig:escopo}.

\begin{figure}
  \centering
  \includegraphics[width=\textwidth]{imagens/Diagrama_Visao_Geral}
  \caption{\it Diagrama de blocos do sistema}
  \label{fig:escopo}
\end{figure}

% TODO explicar como as pessoas ficam amigas na rede social
 O sistema de recomendação utilizará a rede social para obter as informações pessoais e as relações de amizade entre usuários. Neste contexto, um usuário é considerado amigo de outro quando na rede social é estabelecida uma conexão direta entre eles.

 Os usuários podem receber as recomendações dos seus amigos, ou de outras pessoas presentes na rede social. No entanto, as recomendações feitas por amigos serão consideradas, a princípio, mais importantes do que as realizadas por desconhecidos.

% section visao_do_sistema (end)

\section{Descrição do Sistema}

% Retirar informação de cadastro de produtos

 Através do sistema os usuários poderão cadastrar produtos e avaliá-los quantitativamente, sendo possível realizar recomendações a outros usuários. Estes irão fornecer um feedback de tais recomendações, atualizando o valor da reputação do usuário que a forneceu. O sistema de recomendação filtra as informações recebidas por todos os usuários, de modo que apenas aquelas relevantes cheguem ao conhecimento das pessoas. As recomendações geradas pelo sistema são baseadas nas opiniões, nas relações sociais entre os usuários e nas relações encontradas entre os produtos e é o sistema que determina, a partir desses valores, o quão relevante é a recomendação realizada de um usuário para outro. Com base na relação entre usuários, produtos e avaliações o sistema irá inferir as seguintes informações:
 
\begin{itemize}

 \item Similaridade entre usuários

 \item Similaridade entre produtos

 \item Grau de confiança nas recomendações de um usuário para outro

 \item Grau de reputação global de um usuário baseado nas recomendações diretas

 \item Relação de consumo entre diferentes produtos

 \item Recomendações de produtos mais relevantes para um determinado usuário

\end{itemize}

 Para a obtenção das recomendações será feito um estudo para determinar quais critérios inferidos são mais importantes de forma que o sistema apresente uma boa acurácia, abrangência e alta satisfação dos usuários.

% Retirar informação de cadastro de produto

 Ao entrar na rede, o usuário tem a opção de buscar produtos já cadastrados e, caso não existam, cadastrá-los. Neste cadastro, ele pode informar a sua opinião referente ao produto e avaliá-lo, possibilitando ao sistema verificar quais são os seus interesses. Com isso, as pessoas podem recomendar os produtos para seus amigos presentes na rede social, sendo que esta deve ser direcionada para as pessoas que o usuário tenha certo conhecimento que gostarão do produto.

% Explicar melhor a relação de confiança entre usuários

 Os usuários que recebem a recomendação podem acessar o cadastro do produto e avaliá-lo para que o sistema atualize as informações relativas aos seus interesses. Estas informações são utilizadas para verificar a similaridade entre usuários. A avaliação altera o grau de reputação entre o usuário que recomendou o produto e o que recebeu. Caso a avaliação tenha um grau positivo, a reputação do receptor aumenta, porém, caso a avaliação tenha um grau negativo, significa que o usuário que recebeu a recomendação não a aceitou como relevante, fazendo com que o grau de reputação no outro usuário diminua.

 O sistema armazena todas essas informações referentes às avaliações de produtos, confiança e reputação entre usuários para filtrar recomendações entre usuários com baixo grau de confiança. Esse é um dos propósitos do sistema de recomendação: mostrar ao usuário apenas informações relevantes. Assim, caso a pessoa receba recomendações sem conteúdo plausível de um usuário, seu grau de reputação diminui e, quando esta diminuir até um limite, o sistema passa a não mostrar mais recomendações dele em destaque, sendo estas mostradas apenas como uma espécie de spam, mas ainda acessíveis ao usuário que as recebe.

 Também é função do sistema de recomendação realizar recomendações aos usuários para incentivar a avaliação de produtos, pois é assim que as pessoas fornecem informações referentes aos seus interesses. Tais recomendações são relativas aos produtos mais bem avaliados na rede e também aqueles bem avaliados por amigos com alto grau de reputação.

\section{Funcionalidades principais} % (fold)
\label{sec:funcionalidades_principais}

Abaixo estão sumarizadas as principais funcionalidades do sistema acompanhadas das descrições em formato de \textit{user stories}:
% referencia faltando!!!

\begin{itemize}

	\item Recomendação de produtos baseada em preferência do usuário
	\subitem Como usuário, eu quero receber uma lista de produtos que eu provavelemente goste para que eu não precise filtrar os itens que me interessam.

	\item Permitir que o usuário avalie um produto
  \subitem Como usuário, quero fazer a minha avaliação de produtos para que outros tenham conhecimento da minha opinião.

	\item Permitir que o usuário envie uma recomendação a outro usuário
  \subitem Como recomendador, quero enviar uma recomendação de produto para outra pessoa para que ela conheça a minha opinião sobre este item.

% Retirar funcionalidade

	\item Exposição das avaliações em formatos abertos
  \subitem Como agente web, quero obter as avaliações dos usuários em formato padronizado (e aberto) para que possa interpretar estas avaliações.

% Retirar funcionalidade

    \item Cadastro de novos produtos
    \subitem Como usuário, quero poder cadastrar um novo produto que ainda não está cadastrado.
    
% Retirar funcionalidade

    \item Importação de conteúdo em formato aberto
    \subitem Como usuário, quero poder informar uma fonte de conteúdo em formato aberto para que possa adicionar dados externos ao repositório.

    \item Explicação da recomendação
    \subitem Como usuário, quero obter a explicação sobre as recomendações recebidas para analisar criticamente a recomendação e me motivar a avaliar o produto.

    \item Visualização das avaliações recentes dos amigos
    \subitem Como usuário, quero estar ciente sobre as últimas avaliações dos meus amigos para que eu possa saber o que acontece no meu contexto social e conheça novos produtos.

    \item Feedback de recomendação
    \subitem Como recomendador, quero receber feedback sobre as recomendações que enviei para me incentivar a fazer melhores recomendações.

	
\end{itemize}


\section{Requisitos não-funcionais}

Os principais requisitos não-funcionais do sistema são:

\begin{itemize}

    \item Interface web compatível as últimas versões dos \textit{browsers} Internet Explorer, Mozilla Firefox e Safari.
    
    \item \textit{Backend} compatível com servidores Linux.

    \item Tempo médio de resposta menor que 2 segundos para 10 usuários simultâneos quando executado em um servidor com processador Intel Core 2 Duo T7250 ou superior e 2 GB de memória RAM. % esta é a configuração do meu notebook -- Allan

\end{itemize}

\section{Limites do Sistema}

\begin{itemize}
  
    \item O sistema não fará processamento de linguagem natural\footnote{\textit{Natural Language Processing (NLP)}}

\end{itemize}

%\section{Tecnologia} % (fold)
%\label{sec:tecnologia}

% section tecnologia (end)

\section{Implementação} % (fold)
\label{sec:implementacao}


% section implementacao (end)

% chapter especificacao_do_projeto (end)
