\chapter{WEB SEMANTICA} % (fold)
\label{cha:web_semantica}
% Tecnologias e conceitos empregados, contextualização do Projeto de Formatura em sua área de aplicação, revisão da literatura.

A Web é um sistema de documentos em hipermídia que são interligados e executados na Internet. A maioria do conteúdo disponível na Web atualmente é projetado para a leitura por seres humanos, e não possibilitar a utilização por programas de computador de forma significativa.\cite{the_semantic_web} Segundo BERNERS-LEE, a Web Semântica não é uma Web separada, mas um extensão da atual, na qual a informação é disponibilizada com sentido bem definido, aprimorando a capacidade colaboração entre pessoas e computadores.

%The Semantic Web is not a separate Web but an extension of the current one, in which information is given well-defined meaning, better enabling computers and people to work in cooperation.

% 2.	A WS precisará de uma base de dados centralizada para poder raciocinar?

Da mesma forma que os humanos buscam informações em bases de dados descentralizadas na Web, com a adoção da Web Semântica agentes computacionais poderão fazer o mesmo. Essa descentralização de documentos permite que inconsistências ocorram, porém possibilita um rápido crescimento no volume de dados.

% 3.	Explicar a diferença entre apresentar dados e entender dados no contexto da web atual?


\section{Representação de Conhecimento} % (fold)
\label{sec:representação_de_conhecimento}

%4.	Como os atuais sistemas de representação de conhecimento deverão evoluir para a Web semântica?  


%5.	Como se expressa algo usando RDF?

\subsubsection{Ontologia} % (fold)
\label{ssub:ontologia}

%6.	Defina ontologia e mostre um exemplo?

%7.	Em termos físicos, onde ficam as ontologias a serem utilizadas na Web Semântica?

%8.	O que são regras de inferência e como elas se relacionam às ontologias?


% subsubsection ontologia (end)


% section representação_de_conhecimento (end)

\section{Agentes} % (fold)
\label{sec:agentes}

% 9.	Explicar o sentido da frase: “even agents that were not expressly designed to work together can transfer data among themselves when the data come with semantics “

% 10.	Como os agentes poderão compartilhar “provas” das suas afirmações. Ex:?

%11.	Na web semântica existirá a necessidade de informações criptografadas, quais?

%12.	Explicar o sentido da frase: Semantic Web can assist the evolution of human knowledge as a whole.

% section agentes (end)

\section{Arquitetura REST} % (fold)
\label{sec:arquitetura_rest}

% URI

%14.	Forneça exemplos de URIs para: recursos digitais, recursos físicos, conceitos, pessoas, organizações.

%15.	Criar uma URI que identifique você no contexto da WS?

% section arquitetura_rest (end)

% chapter web_semantica (end)