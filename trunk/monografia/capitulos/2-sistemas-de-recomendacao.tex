\chapter{SISTEMAS DE RECOMENDAÇÃO} \pagenumbering{arabic}% (fold)
\label{cha:sistemas_de_recomendação}

\section{Introdução}
Sistemas de recomendação são aqueles que sugerem itens ao seus usuários de forma a ajudá-los a encontrar mais efetivamente os itens de maior interesse dentre uma variedade imensa de opções.

A idéia de recomendar itens é algo que pode ser observado no dia-a-dia das pessoas. Como muitas vezes escolhas precisam ser feitas sem que se tenha uma experiência pessoal das alternativas, as pessoas se baseiam no que as outras dizem sobre o produto ou em resenhas encontradas em revistas e jornais.

Os sistemas de recomendação auxiliam este processo social, agregando opiniões e avaliações de uma comunidade de usuários sobre os produtos e recomendando itens de acordo com o perfil do usuário desta comunidade.

\section{Contexto histórico}
O primeiro sistema de recomendação, Tapestry~\footnote{ver \cite{Goldberg92}}, foi desenvolvido no início da década de 90.~\cite{Resnick97} Na última década o sistemas, principalmente os baseado em filtragem colaborativa, foram um grande foco de estudo~\cite{Herlocker04}.

Graças ao acumulo de informações provenientes da web social, observa-se hoje que vários sistemas de recomendação podem ser experimentados pelo usuários da Internet, como por exemplo:

\begin{itemize}
\item 
A Amazon\footnote{http://www.amazon.com} e o Submarino\footnote{http://www.submarino.com.br}, lojas virtuais de artigos diversos, recomendam itens semelhantes para aqueles usuários que compram um produto ou manifestam interesse em comprá-lo.

\item O Last.fm\footnote{http://last.fm}, uma rede social focada em música, recomenda artistas e canções semelhantes àquelas que os usuários mais gostam de ouvir.

\item O Digg (citar) e o Delicious (citar), sistemas de compartilhamento de links(footnote para bookmark), geram uma lista geral de links recomendados baseado nas opiniões dos usuários do sistema.

\item O StumbleUpon (citar), também um sistema online de compartilhamento de links, permite que os usuários recebam recomendações de links e avaliem se eles gostaram ou não daquele link, gerando recomendações personalizadas baseadas nessa avaliação.
\end{itemize}


\section{Classificações dos sistemas de recomendação}

Nas próximas seções serão apresentadas as diferentes abordagens para se implementar sistemas de recomendação. As primeiras abordagens foram as com filtragem baseada em conteúdo (citar alguém), depois criou-se a filtragem social, utilizando inicialmente a vertente chamada filtragem colaborativa (citar alguém). Vários sistemas são híbridos, isto é, utilizam mais de uma abordagem. Na seção (seção XXX) serão discutidos as diferentes implementações encontradas na literatura.

\subsection{Filtragem baseada em conteúdo} % (fold)
\cite{Balabanovi97}

\subsection{Filtragem social} % (fold)

\subsubsection{Filtragem colaborativa}

\subsubsection{Filtragem baseada em confiança} % (fold)

\section{Discussão sobre as diferentes abordagens}

