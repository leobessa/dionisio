\chapter{SISTEMAS DE RECOMENDAÇÃO} \pagenumbering{arabic}% (fold)
\label{cha:sistemas_de_recomendação}

\section{Introdução}
Sistemas de recomendação~\footnote{Em inglês, \textit{Recommender systems}} são aqueles que sugerem itens ao seus usuários de forma a ajudá-los a encontrar mais efetivamente os itens de maior interesse dentre uma variedade imensa de opções.

A idéia de recomendar itens é algo que pode ser observado no dia-a-dia das pessoas. Como muitas vezes escolhas precisam ser feitas sem que se tenha uma experiência pessoal das alternativas, as pessoas se baseiam no que as outras dizem sobre o produto ou em resenhas encontradas em revistas e jornais.

Os sistemas de recomendação auxiliam este processo social, agregando opiniões e avaliações de uma comunidade de usuários sobre os produtos e recomendando itens de acordo com o perfil do usuário desta comunidade.

\section{Contexto histórico}
O primeiro sistema de recomendação, Tapestry~\footnote{Ver \cite{Goldberg92}}, foi desenvolvido no início da década de 90.~\cite{Resnick97} Na última década o sistemas, principalmente os baseado em filtragem colaborativa, foram um grande foco de estudo~\cite{Herlocker04}.

Graças ao acumulo de informações provenientes da web social, observa-se hoje que vários sistemas de recomendação podem ser experimentados pelo usuários da Internet, como por exemplo:

\begin{itemize}
\item 
A Amazon~\footnote{http://www.amazon.com} e o Submarino~\footnote{http://www.submarino.com.br}, lojas virtuais de artigos diversos, recomendam itens semelhantes para aqueles usuários que compram um produto ou manifestam interesse em comprá-lo.

\item O Last.fm~\footnote{http://last.fm}, uma rede social focada em música, recomenda artistas e canções semelhantes àquelas que os usuários mais gostam de ouvir.

\item O Digg~\footnote{http://digg.com} e o Delicious~\footnote{http://del.icio.us}, sistemas de compartilhamento de links~\footnote{Também conhecidos como bookmarks}, geram uma lista geral de links recomendados baseado nas opiniões dos usuários do sistema.

\item O StumbleUpon~\footnote{http://stumbleupon.com}, também um sistema online de compartilhamento de links, permite que os usuários recebam recomendações de links e avaliem se eles gostaram ou não daquele link, gerando recomendações personalizadas baseadas nessa avaliação.
\end{itemize}


\section{Classificações dos sistemas de recomendação}

Nas próximas seções serão apresentadas as diferentes abordagens para a implementação de sistemas de recomendação. As primeiras abordagens foram a filtragem colaborativa e a filtragem baseada em conteúdo. Vários sistemas são híbridos, isto é, utilizam mais de uma abordagem. Na seção (seção XXX) serão discutidos as diferentes implementações encontradas na literatura.

\subsection{Filtragem baseada em conteúdo} % (fold)
A filtragem baseada em conteúdo consiste na extração de \textit{features} dos itens a serem recomendados e da comparação desses \textit{features} com aqueles que formam o  perfil histórico do usuário. Esse é um dos primeiros métodos que surgiram e sua origem está na comunidade de \textit{information retrieval}.~\cite{Balabanovi97}

Como exemplo, suponha-se que se queira recomendar documentos em formato texto. Os features nesse caso poderiam ser as palavras do texto. O perfil histórico do usuário seria formado pela frequência acumulada das palavras presentes em cada texto avaliado pelo usuário. Um documento neste caso é recomendado se os features (palavras) presentes podem ser encontrados em grande frequência nos documentos avaliados positivamente pelo usuário no passado.

Outros exemplos de features que podem ser usados em um documento são meta-informações como autor, categoria do documento (artigo, jornal, revista, por exemplo), assunto (computação, matemática, artes, esportes), entre outras palavras-chave.

(falar um pouquinho porque é bom/ruim)

\subsection{Filtragem social} % (fold)

% Revisar o seguinte paper e ver se as referencias deles são melhores:                Recommendation as Classi cation: Using Social and Content-Based Information in Recommendation
% Talvez usar o Pazzani96syskill para falar de content-based

A filtragem social\footnote{Termo originado em \cite{Malone87} segundo \cite{Hill95}} consiste em um conjunto de técnicas que utilizam o contexto e as relações sociais de uma comunidade de usuários para fazer recomendações. Ao contrário da filtragem colaborativa, o conteúdo de cada item não é analisado, possibilitando-se recomendar qualquer tipo de item.

\subsubsection{Filtragem colaborativa}

O termo \textit{collaborative filtering} foi cunhado por \cite{Goldberg92}~\footnote{Conforme \cite{Resnick97}}. A abordagem básica consiste em montar um sistema que permite os usuários fazerem avaliações\footnote{Em inglês, \textit{ratings}} dos itens que podem ser recomendados. Isso resultada em uma tripla (usuário, item, avaliação). A partir dessas avaliações, pode-se criar uma lista de recomendação para um determinado usuário. O primeiro passo é encontrar os usuários mais semelhantes a este usuário que vai receber a recomendação. Após esses usuários forem encontrados, a lista de recomendação consistirá dos itens melhor avaliados por cada um deste usuários, sendo o ranking de cada item o produto \textit{rating} do usuário para este item x semelhança deste usuário em relação ao usuário que receberá a recomendação. (transformar esta uma frase em uma expressão matemática)

% Ler e citar \cite{Breese98} para explicar o algoritmo básico de collaborative filtering


\subsubsection{Filtragem baseada em confiança} % (fold)

\section{Discussão sobre as diferentes abordagens}

