\chapter{Instruções Fornecidas aos Participantes} % (fold)
\label{cha:instrucoes_fornecidas_aos_participantes}

Este apêndice apresenta todas as instruções fornecidas aos participantes do experimento. As instruções estão organizadas de acordo com a ordem que foram apresentadas, ou seja, de acordo com a etapa na qual estavam presentes.

\section*{Etapa 1} % (fold)
\label{sec:c_etapa_1}
\begin{itemize}
  \item Olá, <nome do usuário>! Esta é a \textbf{primeira etapa} do nosso experimento de 6 etapas. Abaixo estão listados 20 produtos que você deverá avaliar de acordo com o seu interesse neles. Ressaltamos que este interesse não é apenas um interesse de compra do produto, mas se você acha o produto interessante ou não.
  \item Não ignore as mensagens de popup. Elas lhe questionam se você já ouviu falar do produto avaliado.
  \item Após avaliar todos os 20 produtos, a etapa 2 será carregada automaticamente. Para sua facilidade, um contador de quantos produtos já foram avaliados encontra-se no topo dessa página.
\end{itemize}
% section etapa_1 (end)

% chapter instruçoes_fornecidas_aos_participantes (end)

\section*{Etapa 2} % (fold)
\label{sec:c_etapa_2}
\begin{itemize}
    \item Olá, <nome do usuário>! Esta é a \textbf{segunda etapa} do nosso experimento de 6 etapas. Abaixo está disponível uma busca de produtos cadastrados no sistema.
    \item \textbf{Você deverá procurar 10 produtos ao seu gosto e avaliá-los na mesma forma da etapa 1}, ou seja, de acordo com o seu nível de interesse neles.
    \item Busque os produtos por nome e opcionalmente também por categoria
    \item \textbf{Após avaliar todos os 10 produtos ao seu gosto, a etapa 3 será carregada automaticamente.} Para sua facilidade, um contador de quantos produtos já foram avaliados encontra-se no topo dessa página.
\end{itemize}

% section etapa_2 (end)

\section*{Etapa 3} % (fold)
\label{sec:c_etapa_3}
\begin{itemize}
    \item Olá, <nome do usuário>! Esta é a \textbf{terceira etapa} do nosso experimento de 6 etapas.
    \item \textbf{Abaixo estão listados os amigos do seu grupo. Você deverá recomendar 5 produtos a cada um deles.}
    \item Uma boa recomendação consiste em recomendar produtos que seus amigos tenham muito interesse e que não sejam óbvios a eles.
    \item Para realizar as recomendações para um amigo, \textbf{escolha um deles abaixo.} 
\end{itemize}      

% section etapa_3 (end)

\section*{Etapa 4} % (fold)
\label{sec:c_etapa_4}
\begin{itemize}
  \item Olá, <nome do usuário>! Esta é a \textbf{quarta etapa} do nosso experimento de 6 etapas.
  \item  Abaixo estão listadas algumas pessoas cadastradas no sistema que não fazem parte do seu grupo.
  \item \textbf{Você deverá recomendar um produto a cada um deles.}
  \item Uma boa recomendação consiste em recomendar produtos que as pessoas tenham muito interesse e que não sejam óbvios a eles. 
  \item Para realizar a recomendação a uma pessoa, escolha uma delas abaixo.
\end{itemize}
% section etapa_4 (end)

\section*{Etapa 5} % (fold)
\label{sec:c_etapa_5}
\begin{itemize}
  \item Olá, <nome do usuário>! Esta é a \textbf{quinta etapa} do nosso experimento de 6 etapas.
  \item Abaixo estão listados produtos recomendados a você.
  \item Não esqueça de informar se você conhece ou não o produto
\end{itemize}

\section*{Etapa 6} % (fold)
\label{sec:c_etapa_6}
\begin{itemize}
  \item Olá, <nome do usuário>! Esta é a \textbf{última etapa} do nosso experimento. Abaixo estão listados produtos recomendados a você. Na lista há recomendações realizadas pelo sistema e também tanto por pessoas do seu grupo quanto por outras de outros grupos. Você deverá avaliar estes produtos de acordo com o seu grau de interesse, assim como fez nas primeiras etapas deste experimento.
\end{itemize}
% section etapa_6 (end)

% section etapa_5 (end)