\chapter{ANÁLISES E RESULTADOS} % (fold)
\label{cha:analises_e_resultados}

\section{Introdução}
\label{cha:introducao}

 A análise dos resultados consistiu na elaboração de gráficos e tabelas com os dados retirados da base do experimento realizado. Os seguintes dados foram considerados para a análise:
 
\begin{itemize}
	\item Grupos de usuários
	
	Para se ter noção dos laços de amizades entre os participantes cadastrados no experimento. São 12 grupos no total, sendo que cada um deles possui no máximo 5 pessoas.
	
	\item Categorias de produtos
	
	As categorias de produtos foram consideradas na análise para discretizar as avaliações das recomendações por tipos de produtos.
	
	\item Recomendações realizadas por amigos
	
	Informações dos produtos que foram recomendados na etapa 3 do experimento, onde os participantes deveriam recomendar 5 produtos a cada amigo presente no seu grupo. Cada participante possui no máximo 20 recomendações, sendo que alguns possuem apenas 15 devido à desistência de participantes do seu grupo.
	
	\item Recomendações realizadas por desconhecidos
	
	Informações dos produtos que foram recomendados na etapa 4 do experimento, onde os participantes deveriam recomendar 1 produto a alguns participantes de diferentes grupos, considerados desconhecidos. Cada participante possui no máximo 10 recomendações realizadas por desconhecidos, sendo que alguns possuem menos de 10 recomendações devido à desistência de participantes no experimento.
	
	\item Recomendações realizadas pelo sistema
	
	Todas as recomendações que o sistema realizou para os participantes. Contém a informação de qual participante recebeu a recomendação e qual foi o produto recomendado.
	
	\item Avaliação prevista do produto pelo sistema
	
	Ao recomendar um produto a um participante, o sistema calcula uma nota prevista para o mesmo. Essas informações foram armazenadas e consideradas durante a análise e exposição dos dados do experimento.
	
	\item Avaliações dos produtos
	
	Avaliações de todos as avaliações de produtos no sistema. Contém os produtos avaliados na duas primeiras etapas do experimento, quando os participantes avaliaram 20 produtos em comum e 10 produtos de seu interesse, e as avaliações de produtos recomendados tanto pelos participantes como pelo sistema.
	
	\item Algoritmo utilizado para a recomendação
	
	Dentre as recomendações realizadas pelo sistema, foi retirada da base de dados a informação de qual algoritmo foi utilizado para gerar a recomendação.
	
\end{itemize}

 De posse dos dados, foram considerados 3 tipos de cenários possíveis para a análise dos mesmos:
 
 \begin{itemize}
	\item Cenário global
	
	Foram analisados os dados de todos os participantes presentes no experimento, sem discriminar os dados pelos grupos de amigos.
	
	\item Grupos de amigos
	
	Os dados foram analisados grupo a grupo para comparar os resultados obtidos apenas participantes amigos.
	
	\item Categorias de produtos
	
	%TODO
	A categoria de produtos 
	
	\item Algoritmos de recomendação
	
	A discriminação entre os diferentes tipos de algoritmos de recomendação serviram para comparar o algoritmo baseado em confiança proposto pelo estudo frente aos algoritmos já existentes baseados em similaridade de perfis e produtos.
	
\end{itemize}

 Em cada cenário foram analisadas as seguintes possíveis medidas:
 
\begin{itemize}
	\item Erro na avaliação do produto
	\item Taxa de rejeição das recomendações
	\item Taxa de serendipidade
	\item Grau de confiança calculado
\end{itemize}

 Cada medida e cenário considerado será discutido nas seguintes seções.
 
\section{Análise de Desempenho}


\section{Análise Comparativa dos Algoritmos de Recomendação}

\subsection{Recomendações do Sistema}

\subsection{Recomendações Diretas}


\section{Análise de Rejeição das Recomendações}

\subsection{Recomendações do Sistema}

\subsection{Recomendações Diretas}

\section{Análise do Algoritmo Baseado em Confiança}
